\documentclass{article}

\usepackage[preprint]{nips_2018}
\usepackage[pdftex]{graphicx}
\usepackage[utf8]{inputenc}  
\usepackage[T1]{fontenc}     
\usepackage{hyperref}        
\usepackage{url}             
\usepackage{booktabs} 
\usepackage{amsfonts} 
\usepackage{nicefrac}        
\usepackage{microtype}       
\usepackage{bookmark}
\usepackage{array}

\usepackage{amsmath}
\usepackage{amsfonts} 
\usepackage{amssymb}
\usepackage{array}
\usepackage{algorithmic}
\usepackage{algorithm}


\title{Distributed Operation and Control in Power Systems}

\author{
  Draft \\
}

\begin{document}

\maketitle

\begin{abstract}


\end{abstract}

%\tableofcontents
\section{Lecture 1}
\subsection{Aims}

\subsection{Foundations}

\subsection{Possible challenges}
\begin{itemize}
  \item High penetration renewables in power system requires for new methods and technology for the stability to bulk grid, management, real-time calculation, low transmitting latency (0-1 bit transmitting),  
  \item AC-DC converters
  \item Microgrid
  \item Different / multi- time scales for power system operation: 1) controlling perspective: transient stability; 2) operation and energy management: steady state, economics.
  \item What is and how to build and describe uncertainty ? Robust: worst scenarios, featuring computing efficiency, conservative; stochastic: expected cost function of different scenarios, less conservative; MPC: several steps ahead, frequency control and voltage control of system. 
  \item Where is the uncertainty from: market prices, renewables - > Gaussian distribution, prediction errors to decision error, end-to-end model???.
  \item Several constraints: supply demand curves, congestions, power loss.
  \item For different uncertainty sources, there may be different uncertainty processing way. e.g. renewables: worst, prices: new.
  \item Robust security constraints: get some convex approximation of current problems given the new components of the system;
  \item what is the resilient grid ?
  \item Downward and upward flexibility. 
  \item Grid-connected entity, grid-forming converters;
  \item cyber-physical system (CPS theoretic foundations' system ?)
  \item How to calculate the gap between learning and optimizing ?
  \item Carbon footprint calculation !!!
\end{itemize}

Generation side:

What is the flexibility from the system ? This can be categorized into two types:
\begin{itemize}
  \item Congestion;
  \item Reserve;
  \item Virtual inertia;
  \item Power quality/ harmonics;
  \item Black out;
  \item Peak shaving;
  \item Frequency regulation (Primary secondary, tertia);  
  \item Power factor correction; 
\end{itemize}

\begin{itemize}
  \item VCC;
  \item Power factor;
  \item Active distribution networks;
  \item High penetration of high renewables and power electronics;
  \item Micro-grid; 
  \item pico grid;
  \item vein system.
  \item MPT plot figures;
  \item Transition state;
  \item Distinction between power density and energy density.
\end{itemize}

$$
H \frac{d \Delta f}{d t} = D \Delta f  - \sum_{j\in i} P_{ij} + P(t) + U 
$$

Network side:

Demand side: 


\subsection{Possible solution by distributed algorithm}

\subsection{Power flow of the system}

\section{Lecture 2}

\subsection{Possible distributed algorithm}

1. Online distributed learning algorithm considering the convergence speed the conventional iterative / consensus-based methods.

2. Learn from sparse communication networks and the sparse vunlary analysis of the network. 

3. The calculation of LODF and SF is quite time consuming which can further embeds some machine learning techniques.
\section*{References}

\medskip

\small



\end{document}