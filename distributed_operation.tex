\documentclass{article}

\usepackage[utf8]{inputenc}
\usepackage[T1]{fontenc}
\usepackage[preprint]{nips_2018}
\usepackage[utf8]{inputenc}  
\usepackage[T1]{fontenc}     
\usepackage{hyperref}        
\usepackage{url}             
\usepackage{booktabs} 
\usepackage{amsfonts} 
\usepackage{nicefrac}        
\usepackage{microtype}       
\usepackage{bookmark}
\usepackage{array}
\usepackage{color}
\usepackage{amsmath}
\usepackage{amsfonts} 
\usepackage{amssymb}
\usepackage{array}
\usepackage{algorithmic}
\usepackage{algorithm}

\newcommand{\slw}{\color{red}}
\newcommand{\insight}{\color{blue}}


\title{Distributed Operation and Control in Power Systems}

\author{
  Class document\\
}

\begin{document}

\maketitle

\begin{abstract}

\end{abstract}

%\tableofcontents
\section{Lecture 1}
\subsection{Aims}

\subsection{Foundations}

\subsection{Possible challenges}
\begin{enumerate}
  \item High penetration renewables in power system requires for new methods and technology for the stability to bulk grid, management, real-time calculation, low transmitting latency (0-1 bit transmitting),  
  \item AC-DC converters
  \item Microgrid
  \item Different / multi- time scales for power system operation: 1) controlling perspective: transient stability; 2) operation and energy management: steady state, economics.
  \item What is and how to build and describe uncertainty ? Robust: worst scenarios, featuring computing efficiency, conservative; stochastic: expected cost function of different scenarios, less conservative; MPC: several steps ahead, frequency control and voltage control of system. 
  \item Where is the uncertainty from: market prices, renewables -> Gaussian distribution, prediction errors to decision error, end-to-end model???.
  \item Several constraints: supply demand curves, congestions, power loss.
  \item For different uncertainty sources, there may be different uncertainty processing way. e.g. renewables: worst, prices: new.
  \item Robust security constraints: get some convex approximation of current problems given the new components of the system;
  \item what is the resilient grid ?
  \item Downward and upward flexibility. 
  \item Grid-connected entity, grid-forming converters;
  \item cyber-physical system (CPS theoretic foundations' system ?)
  \item How to calculate the gap between learning and optimizing ?
  \item Carbon footprint calculation !!!
\end{enumerate}

Generation side:

What is the flexibility from the system ? This can be categorized into two types:
Midterm !!!!
\begin{enumerate}
  \item Congestion;
  \item Reserve;
  \item Virtual inertia;
  \item Power quality/ harmonics;
  \item Black out;
  \item Peak shaving;
  \item Frequency regulation (Primary secondary, tertiary);  
  \item Power factor correction; 
  \item Ramping (eclipse condition);
\end{enumerate}

\begin{enumerate}
  \item VCC;
  \item Power factor;
  \item Active distribution networks;
  \item High penetration of high renewables and power electronics;
  \item Micro-grid; 
  \item pico grid;
  \item vein system.
  \item MPT plot figures;
  \item Transition state;
  \item Distinction between power density and energy density.
\end{enumerate}

$$
H \frac{d \Delta f}{d t} = D \Delta f  - \sum_{j\in i} P_{ij} + P(t) + U 
$$

Network side:

Demand side: 


\subsection{Possible solution by distributed algorithm}

\subsection{Power flow of the system}

\newpage

\section{Lecture 2}

\subsection{Possible distributed algorithm challenges}

1. Online distributed learning algorithm considering the convergence speed the conventional iterative / consensus-based methods.

2. Learn from sparse communication networks and the fast sparse vunerality analysis of the network. 

3. The calculation of LODF and shift factors is quite time-consuming which can further embeds some machine learning sparse matrix techniques.

\begin{enumerate}
  \item Robust; 
  \item Stochastic;
  \item MPC (applied for proposed methods) Distributionally robust optimization;
\end{enumerate}

The formulation of load information is related the node voltage and frequency.

Two concepts of N-1 criteria: 1) corrective; 2) preventive for ship board power system.

Grid-tied formulation;
\begin{enumerate}
  \item Current sources: P and Q; 
  \item Voltage sources: V and S;
\end{enumerate}

Inertia related:
\begin{enumerate}
  \item RoCof;
  \item Nadir;
  \item Steady frequency deviation;
\end{enumerate}

Challenges for renewables: 

\begin{enumerate}
  \item low inertia;
  \item intermittence / fluctuation;
  \item harmonics;
  \item uncertainty;
  \item non-linearity;
  \item lower market prices: power prices and capacity prices;
  \item ramping reserve;
\end{enumerate}

Frequency droop control: proportional and integral control, (PI control) steady state errors; 

Structure of networks: mesh or ring networks.


Global control: 1) frequency; 2) reserve; 3) virtual inertia; 4) ramping; 5) blackout start; 6) peak-shaving.

Locational control: 1) congestion; 2) voltage; 3) harmonics; 

Distance between generation and load for reducing the power loss: local balance.

50\% by 2050

Price time: Time of use (ToU) price.

EV fleet: aggregation of EV devices.

Power networks and gas networks to lower the carbon emission.

Cyber physical power Systems

\newpage
\section{Lecture 3}

Background: stop building fossil generators. 

Discriminate the need of power demand and energy demand in power system, power density. 

The location of wind rich and load dense area is not the same, which means the disparity of energy system.

The distribution of wind resources and solar with the potential of renewables is spatial and temporal. 

Features of wind power:
\begin{enumerate}
  \item Effect of a plate on a wind flow;
  \item Wellbull distribution;
  \item MPPT;
  \item Inertia from the wind turbines {\slw inertia deduction from the controlling policy, from the perspective of SCUC};
  \item Higher total capacity;
  \item Virtual inertia control policy and its trading, marketing mechanism design to incentivize the participation;
  \item Swing equation from the systems;
  \item Coupled system and decoupled systems of dynamics frequency analysis under the context of AC-DC-AC formulation;
  \item Security analysis;
  \item Economics analysis;
  \item The number of blades in the wind turbines: aerodynamic analysis with changing of the blades of wind turbines to get the critical point between 3 and 4;
  \item The places of wind turbines can be located in off-shore;
  \item Output curve of wind power output with the relationship between the wind speed and wind power output;
  \item Maximum peak power tracking (MPPT);
  \item PMSG control;
  \item DFIG control policy, park transformation;
  \item Grid forming system: voltage source based inverter, voltage reference voltage amp. \& freq. (weak grid and bulk grid)
  \item Grid feeding control: current-source based inverter, current reference, active \& reactive power;
  \item Virtual inertia emulation support system;
  \item Variable-speed wind turbine with hydrogen storage system;
\end{enumerate}

$$S^2 = P^2 + Q^2$$

$$\frac{1}{1+\tau s}$$

HVDC system formulation:

\begin{enumerate}
  \item Lumped model;
  \item Long distance >= 100 km and power 200-900 MW;
  \item DC: frequency;
  \item features of HVDC transmission;
  \item vertical design system;
  \item of the wind turbine design
\end{enumerate}

Some discussion, what is the problem with rooftop WT ? (Midterm test)

Solar power features:
\begin{enumerate}
  \item Imputed silicon;
  \item Silicon is 2$^{nd}$ abundant element on earth;
  \item Conductor, semi-conductor, insulator;
\end{enumerate}

PV / WT problem for more stringent problems by shaping the peak here:
\begin{enumerate}
  \item congestions;
  \item over voltages;
  \item Power ramping control (ramping constraints);
  \item Curtailed power generation;
  \item Delta power reserve control;
\end{enumerate}

Modifying MPPT algorithm for PV and WT; 
Comparison clear day and cloudy day.

Ramping control of the systems;

Power reserve from the renewable energies for power reserve control {\slw turning the MPPT control to reserve control for participating different markets(energy market, reserve market) and some penalty inclusion system}; 

Offset prediction errors and participate in reserve market;

Delta power reserve control: 

Feasible power region: 

Reverse / inverse probation: 

Control requirements and features for PV systems:
\begin{enumerate}
  \item PV panels
  \item Fault ride-through;
  \item Pros and cons of PV power;
  \item Energy storage system;
  \item Uncertain system;
  \item Solution of the system for green house emission;
  \item How to recycle the PV panels in the system;
  \item {\slw further consider the accurate control policy of renewable energies}
  \item {\slw sizing and siting the PV and wind turbines to achieve ...}
\end{enumerate}

\section{Lecture 4}

Diversify energy sources;

Some discussion on EV:
\begin{enumerate}
  \item V2G;
  \item G2V;
  \item {\slw V2h}: utilize vehicle to support the home energy management;
  \item Price-incentivized method;
  \item {\slw Different price means different states of the system};
  \item How to allocate the number/capacity of V2G, G2V, V2h for smart charging/discharging for energy arbitrage?
  \item First we need to accumulate the info.
  \item Final project on this topic to solve the power outage in the {\slw V2h} scenario. 
\end{enumerate}

Target of future power grid:
\begin{enumerate}
  \item High flexibility;
  \item High efficiency;
  \item Modular components: Plug-in and play system: {\slw plug-in and play learn-to-optimize};
  \item Demand response;
  \item Integrated energy system;
  \item Improved dynamics performance;
  \item Reliable optimization;
  \item Scale and operation rules;
\end{enumerate}

\section{Lecture 5}

VPP ignores the topology of the systems, which provide the service based on this feature.

\textbf{Wake the sleeping resources in the energy system.}

Virtual hydrogen power.

VPP provides a possible initial optimal value for AC power flow.

\section{Lecture 6}

$$
f_i = a_i P_i^2 + b_i P_i + c_i \\
\text{Reserve requirements:} R_i \\
\text{Probability:} \gamma \\
$$

\begin{eqnarray}
  \begin{aligned}
    f_i = a_i P_i^2 + b_i P_i + c_i \\ 
    \text{Reserve requirements:} R_i \\
    \text{Probability:} \gamma_i \\
  \mathbf{P}(P_{max} - P_i \geq R_i) \geq \gamma_i \\
  \end{aligned}
\end{eqnarray}

\section{Lecture 7}

Transmission networks: R $<<$ X;

Distribution networks: R $\approx $ X;

Microgrid: R $$>>$$ X;

$$
\Phi(x) = \sum_{t=1}^T []
$$

What is the influence of massive DER in power system ?

AC power flow needs some approximation.

Fast Frequency Regulation.

Physical level of system.

Local control of system.

Power factor correction.

Frequency, voltage fluctuation.

Harmonics, distortion, low-inertia, load profile distortion, bidirectional (reverse) powerflow, excess power loss,

Low inertia (stability), over loading problem.

EV problem

Prediction problems (load shedding, day ahead scheduling.)

N-1 security constraints.

Upstream networks.

Microgrid control architecture. 

Why step-response not sine curve ?
Because the step-response including a high spectrum of sine wave.


\section{Lecture 8}



\section*{References}

\medskip

\small



\end{document}